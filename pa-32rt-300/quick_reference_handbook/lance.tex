\documentclass{article}
\usepackage{geometry}
    \geometry{
        a5paper,
        portrait,
        % The Jeppesen plate appears to be closer to 0.25in. I think
        % that 0.5in is looking best for checklists.
        margin=0.5in,
        % I decided arbitrarily on these values to maximize margins,
        % headers, and content in one page (and satisfy fancyhdr).
        % https://tex.stackexchange.com/questions/132170/what-do-headheight-headsep-etc-do-in-the-vmargin-package
        headsep=12pt,
        footskip=12pt,
        includehead,
        includefoot
    }
% NASA says:
% > The horizontal spacing between characters should be 25% of the
% > overall size and not less than one stroke width.
%
% The microtype documentation says:
% > Letterspaced fonts for which settings don’t exist will be spaced out
% > by the default of 0.1 em [...]
% AND
% > The amount is specified in thousandths of 1 em [...]
%
% So, we're scaling the default spacing by 25%, and then converting to
% housandths of an em (0.1 * 1000 * .25).
\usepackage[letterspace=25]{microtype}
% Used for the checklist frames.
\usepackage{tcolorbox}
% For the preflight checklist square.
\usepackage{amssymb}
% For finer control over multi-column layouts.
\usepackage{multicol}
% For finer control over headers and footers.
\usepackage{fancyhdr}

% This is the box that surrounds the checklist items.
\newtcolorbox{checklist}[1]{
    colback=white,
    colframe=black,
    fonttitle=\centering\bfseries,
    adjusted title={#1},
    sharpish corners
}

\newtcolorbox{checklist_emerg}[1]{
    coltitle=red!25!black,
    colback=white,
    colframe=red!75,
    fonttitle=\centering\bfseries,
    adjusted title={#1},
    sharpish corners
}

% This is a macro that formats the checklist items and adds a new line.
\def\checkitem#1#2{
    #1\dotfill#2

}

% This is a macro to assist in the preflight checklist.
\def\todoitem#1{
    \item[$\square$] #1 \dotfill
}

% Set the default font family to sans-serif.
\renewcommand{\familydefault}{\sfdefault}

% NASA says:
% > The vertical spacing between lines should not be smaller than 25-33%
% > of the overall size of the font.
\renewcommand{\baselinestretch}{1.25}

% Configure the header and footer.
\pagestyle{fancy}
\fancyhf{}
\lhead{Quick Reference Handbook}
\rhead{PA-32RT-300, Lance II}
\rfoot{\thepage}

\begin{document}

% Apply microtype tracking adjustments.
\lsstyle

% [...] you can say \raggedcolumns if you don’t want the bottom lines to
% be aligned. The default is \flushcolumns, so TEX will normally try to
% make both the top and bottom baselines of all columns align.
\raggedcolumns

\section{Preflight Checklist}

\subsection{Master Switch On}

\begin{itemize}
    \todoitem{Interior and exterior lights}
    \todoitem{Stall warning horn}
    \todoitem{Pitot heat}
\end{itemize}

\subsection{Walk Around}

\begin{itemize}
    \todoitem{Control surfaces and cables}
    \todoitem{Drain fuel tank sumps}
    \begin{itemize}
        \item[$\bullet$] Remove all water and sediment; verify proper fuel.
    \end{itemize}
    \todoitem{Drain fuel tank strainer}
    \begin{itemize}
        \item[$\bullet$] {
            Move the selector to the off position, left, then right,
            while draining the strainer sump.
        }
    \end{itemize}
    \todoitem{Propeller}
    \todoitem{Air inlets and alternator belt tension}
    \todoitem{Oil level (10-12 quarts)}
    \todoitem{No obvious oil or fuel leaks}
\end{itemize}

\subsection{Landing Gear}

\begin{itemize}
    \todoitem{Strut exposure ($\geq 4.0''$ for main, $\geq 2.60''$ for nose)}
    \todoitem{Visual inspection of tires}
    \todoitem{Visual inspection of brake blocks}
\end{itemize}

% \subsection{Finishing Up}

% \begin{itemize}
%     \todoitem{Clean windshield}
%     \todoitem{Prep cockpit}
%     \todoitem{Engine times and Stratus}
% \end{itemize}

\section{Normal Procedures}
\clearpage

\twocolumn

\begin{checklist}{Before Start}
    \checkitem{Walk Around}{Completed}
    \checkitem{Seat Belts}{Fastened}
    \checkitem{PIC}{Established}
    \checkitem{Passengers}{Briefed}
\end{checklist}

\begin{checklist}{Engine Start}
    \checkitem{Parking Brake}{Set}
    \checkitem{Fuel}{Desired Tank}
    \checkitem{Alternate Air}{Off}
    \checkitem{Master Switch}{On}
    \checkitem{3-in-the-green}{Check}
    \checkitem{Fuel Pump}{On}
    \checkitem{Throttle}{$\frac{1}{4}''$}
    \checkitem{Mixture}{Prime}
    \checkitem{CLEAR/Starter}{Engage}
    \checkitem{Throttle}{Idle}
    \checkitem{Mixture}{Lean for Taxi}
    \checkitem{Oil Pressure}{Check}
    \checkitem{Load Meter}{Check}
    \checkitem{Fuel Pump}{Off}
    \checkitem{Fuel Pressure}{Check}
\end{checklist}

\begin{checklist}{After Start}
    \checkitem{Avionics Master}{On}
    \checkitem{Circuit Breakers}{Check}
    \checkitem{Garmin Database}{Check}
    \checkitem{Garmin Self-Test}{Check}
\end{checklist}

\begin{checklist}{Before Taxi}
    \checkitem{Transponder}{Set}
    \checkitem{COM/NAV}{Set}
    \checkitem{Initial Altitude}{Set}
    \checkitem{Initial Heading}{Set}
\end{checklist}

\begin{checklist}{Taxi}
    \checkitem{Exterior Lights}{Set}
    \checkitem{Brakes}{Check}
    \checkitem{Heading Indicator}{±5°}
    \checkitem{Attitude Indicator}{Check}
    \checkitem{Turn Coordinator}{Check}
\end{checklist}

\begin{checklist}{Engine Run-Up}
    \checkitem{Mixture}{Full Rich}
    \checkitem{Prop}{Full Forward}
    \checkitem{Throttle}{2000 RPM}
    \checkitem{Mags}{Check L \& R}
    \centering{(max drop 175; max $\Delta$ 50)}
    \checkitem{Alternate Air}{Check}
    \checkitem{Prop}{Cycle}
    \centering{
        ($\downarrow$ rpm $\uparrow$ mp $\downarrow$ oil press.)
    }
    \checkitem{Vacuum}{4.9-5.1$''$Hg}
    \checkitem{Load Meter}{Check}
    \checkitem{Fuel Pressure}{Check}
    \checkitem{Oil Pressure/Temp}{Check}
    \checkitem{Alternate Static}{Check}
    \checkitem{Annunciator Panel}{Check}
    \checkitem{Throttle}{Idle}
    \checkitem{Mixture}{Lean for Taxi}
\end{checklist}

\begin{checklist}{Before Take Off}
    \checkitem{Flight Controls}{Check}
    \checkitem{Flight Instruments}{Check}
    \checkitem{Alternate Air}{Off}
    \checkitem{Flaps}{Set}
    \checkitem{Trim}{Set}

    \hfill

    \centering{
        \emph{Take Off Briefing}
    }

    \hfill

    \checkitem{Take Off Distance}{Briefed}
    \checkitem{Take Off Minimums}{Briefed}
    \checkitem{Departure Procedure}{Briefed}
    \checkitem{Lost Comms}{Briefed}

    \hfill

    \centering{
        \emph{Entering Runway}
    }

    \hfill

    \checkitem{Time Off}{Noted}
    \checkitem{Doors/Windows}{Closed}

    \hfill

    \checkitem{Exterior Lights}{Set}
    \checkitem{Fuel Pump}{On}
    \checkitem{Mixture}{Full Rich}
    \checkitem{Prop}{Full Forward}
\end{checklist}

\begin{checklist}{After Take Off}
    \checkitem{Landing Gear}{Up}
    \checkitem{Prop}{-100 RPM}
    \checkitem{Fuel Pump}{Off}
    \checkitem{Mixture}{Constant EGT}
\end{checklist}

\begin{checklist}{Approach and Landing}
    \checkitem{Altimeter}{Set}
    \checkitem{DA or MDA (MSL)}{Set}

    \hfill

    \checkitem{Airspeed}{$<$ 129 KIAS}
    \checkitem{Landing Gear}{Down}
    \checkitem{Throttle}{17$''$}
    \checkitem{Prop}{2400 RPM}
    \checkitem{Mixture}{Constant EGT}
    \checkitem{Airspeed}{100 KIAS}
\end{checklist}

\begin{checklist}{Go Around}
    \checkitem{Fuel Pump}{On}
    \checkitem{Mixture}{Full Rich}
    \checkitem{Prop}{Full Forward}
    \checkitem{Throttle}{Full Power}
    \checkitem{Landing Gear}{Up}
\end{checklist}

\begin{checklist}{V-Speeds}
    \checkitem{$V_{A}$}{112-132 KIAS}
    \checkitem{$V_{LE}$}{129 KIAS}
    \checkitem{$V_{LO}$}{106 KIAS}
    \checkitem{$V_{X}$}{68 / 87 KIAS}
    \checkitem{$V_{Y}$}{87 / 92 KIAS}
    \checkitem{$V_{CC}$}{104 KIAS}
    \checkitem{$V_{REF}$ (Normal)}{95 KIAS}
    \checkitem{$V_{REF}$ (Short Field)}{75 KIAS}
\end{checklist}

\onecolumn

\begin{checklist_emerg}{Electrical Fire (Smoke in Cabin)}
    \begin{enumerate}
        \item \checkitem{Master switch}{off}
        \item \checkitem{Avionics master}{off}
        \item \checkitem{Electrical switches}{off}
        \item{
            \textbf{If no smoke:}
            \begin{enumerate}
                \item \checkitem{Circuit breakers}{note tripped}
                \item \checkitem{Circuit breakers}{off}
                \item \checkitem{Master switch}{on}
                \item{
                    \textbf{If no smoke:}
                    \begin{enumerate}
                        \item \checkitem{Avionics master}{on}

                    \end{enumerate}
                }
            \end{enumerate}
        }
    \end{enumerate}
\end{checklist_emerg}

\begin{checklist_emerg}{Alternator Failure}
    \begin{enumerate}
        \item Verify failure.
        \item Reduce electrical load as much as possible.
        \item \checkitem{Alternator circuit breakers}{check}
        \item \checkitem{Alt switch}
                {off (for 1 second), then on}
        \item \textbf{If no output:}
        \begin{enumerate}
            \item \checkitem{Alt switch}{off}
        \end{enumerate}
        Reduce electrical load and land as soon as practical.


        If battery is fully discharged, the gear will have to be lowered
        using the emergency gear extension procedure. Position lights 
        will not illuminate.
    \end{enumerate}
\end{checklist_emerg}

\begin{checklist_emerg}{Emergency Landing Gear Extension}
    \begin{enumerate}
        \item \checkitem{Master switch}{check on}
        \item \checkitem{Circuit breakers}{check}
        \item \checkitem{Radio lights}{off (in daytime)}
        \item{
            \textbf{If landing gear does not check down and locked:}
            \begin{enumerate}
                \item \checkitem{Airspeed}{below 87 KIAS}
                \item \checkitem{Landing gear selector}{down}
                \item \textbf{If landing gear still does not check down and locked:}
                \begin{enumerate}
                    \item \checkitem{Emergency gear lever}{emergency down}
                    \raggedleft{(while fishtailing airplane)}
                \end{enumerate}
            \end{enumerate}
        }

    \end{enumerate}
\end{checklist_emerg}


\end{document}
